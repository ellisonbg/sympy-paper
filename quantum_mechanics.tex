SymPy has extensive capabilities for symbolic quantum mechanics in the
\verb|sympy.physics.quantum| subpackage. At the base level, this subpackage has
Python objects to represent the different mathematical objects relevant in
quantum theory [1]: states (bras and kets), operators (unitary, hermitian,
etc.) and basis sets as well as operations on these objects such as tensor
products, inner products, outer products, commutators, anticommutators, etc.
The base objects are designed in the most general way possible to enable any
particular quantum system to be implemented by subclassing the base operators
to provide system specific logic.

The `quantum` subpackage has a general purpose `qapply` function that is
capable of applying operators to states symbolically as well as simplifying a
wide range of symbolic expressions involving different types of products and
commutator/anticommutators. The state and operator objects also have a rich API
for declaring their representation in a particular basis. This includes the
ability to specify a basis for a multidimensional system using a complete set
of commuting Hermitian operators.

On top of this base set of objects, a number of specific quantum systems have
been implemented. First, we have implemented the traditional algebra for
quantum angular momentum [2]. This allows the different spin operators ($S_x$,
$S_y$, $S_z$) and their eigenstates to be represented in any basis and for any
spin quantum number. Facilities for Clebsch-Gordan Coefficients, Wigner
Coefficients, rotations, and angular momentum coupling are also present in
their symbolic and numerical forms.

Second we have implemented a full set of states and operators for symbolic
quantum computing. Multidimensional qubit states can be represented
symbolically and as vectors. A full set of one ($X$, $Y$, $Z$, $H$, etc.) and
two qubit ($CNOT$, etc.) gates (unitary operators) are provided. These can be
represented as matricies (sparse or dense) or made to act on qubits
symbolically without representation. With these gates, it is possible to
implement a number of basic quantum circuits including the quantum Fourier
transform, quantum error correction, quantum teleportation, Grover's algorithm,
dense coding, etc.

Other examples of particular quantum systems that are implemented in SymPy
include second quantization, the simple harmonic oscillator (position/momentum
and raising/lowering forms) and continuous position/momentum based systems.

The package also contains exact symbolic energies and wave functions of several
simple systems like the Hydrogen atom (non-relativistic and relativistic) and
harmonic oscillator (1d and spherical 3D).

1. J.J. Sakurai and J.J. Napolitano, "Modern Quantum Mechanics," Addison-Wesley
(2010).
2. R.N. Zare, "Angular Momentum: Understanding Spatial Aspects in Chemistry and
Physics," Wiley (1991).
3. M.A. Nielsen and I.L. Chuang, "Quantum Computation and Quantum Information,"
Cambridge University Press (2011).

