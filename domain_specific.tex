SymPy includes several packages that allow users to solve domain specific
problems. For example, a comprehensive physics package is included that is
useful for solving problems in classical mechanics, optics, and quantum
mechanics along with support for manipuating physical quantities with units.

\subsection{Vector Algebra}

The \verb|sympy.physics.vector| package provides reference frame, time, and
space aware vector and dyadic objects that allow for three dimensional
operations such as addition, subtraction, scalar multiplication, inner and
outer products, cross products, etc. Both of these objects can be written in
very compact notation that make it easy to express the vectors and dyadics in
terms of multiple reference frames with arbitrarily defined relative
orientations. The vectors are used to specify the positions, velocities, and
accelerations of points, orientations, angular velocities, and angular
accelerations of reference frames, and force and torques. The dyadics are
essentially reference frame aware $3 \times 3$ tensors. The vector and dyadic
objects can be used for any one-, two-, or three-dimensional vector algebra and
they provide a strong framework for building physics and engineering tools.

The following Python interpreter session showing how a vector is created using
the orthogonal unit vectors of three reference frames that are oriented with
respect to each other and the result of expressing the vector in the $A$
frame. The $B$ frame is oriented with respect to the $A$ frame using Z-X-Z
Euler Angles of magnitude $\pi$, $\frac{\pi}{2}$, and
$\frac{\pi}{3}$\si{\radian}, respectively whereas the $C$ frame is oriented
with respect to the $B$ frame through a simple rotation about the $B$ frame's
X unit vector through $\frac{\pi}{2}$\si{\radian}.

\begin{verbatim}
>>> from sympy import pi
>>> from sympy.physics.vector import ReferenceFrame
>>> A = ReferenceFrame('A')
>>> B = ReferenceFrame('B')
>>> C = ReferenceFrame('C')
>>> B.orient(A, 'body', (pi, pi / 3, pi / 4), 'zxz')
>>> C.orient(B, 'axis', (pi / 2, B.x))
>>> v = 1 * A.x + 2 * B.z + 3 * C.y
>>> v
A.x + 2*B.z + 3*C.y
>>> v.express(A)
A.x + 5*sqrt(3)/2*A.y + 5/2*A.z
\end{verbatim}

\subsection{Classical Mechanics}

The \verb|physics.mechanics| package utilizes the \verb|physics.vector| package
to populate time aware particle and rigid body objects to fully describe the
kinematics and kinetics of a rigid multi-body system. These objects store all
of the information needed to derive the ordinary differential or differential
algebraic equations that govern the motion of the system, i.e., the equations
of motion. These equations of motion abide by Newton's laws of motion and can
handle any arbitrary kinematical constraints or complex loads. The package
offers two automated methods for formulating the equations of motion based on
Lagrangian Dynamics~\cite{Lagrange1811} and Kane's Method~\cite{Kane1985}. Lastly, there
are automated linearization routines for constrained dynamical
systems based on~\cite{Peterson2014}.

\subsection{Symbolic Quantum Mechanics}

SymPy has extensive capabilities for symbolic quantum mechanics in the
\verb|sympy.physics.quantum| subpackage. At the base level, this subpackage has
Python objects to represent the different mathematical objects relevant in
quantum theory [1]: states (bras and kets), operators (unitary, hermitian,
etc.) and basis sets as well as operations on these objects such as tensor
products, inner products, outer products, commutators, anticommutators, etc.
The base objects are designed in the most general way possible to enable any
particular quantum system to be implemented by subclassing the base operators
to provide system specific logic.

The `quantum` subpackage has a general purpose `qapply` function that is
capable of applying operators to states symbolically as well as simplifying a
wide range of symbolic expressions involving different types of products and
commutator/anticommutators. The state and operator objects also have a rich API
for declaring their representation in a particular basis. This includes the
ability to specify a basis for a multidimensional system using a complete set
of commuting Hermitian operators.

On top of this base set of objects, a number of specific quantum systems have
been implemented. First, we have implemented the traditional algebra for
quantum angular momentum [2]. This allows the different spin operators ($S_x$,
$S_y$, $S_z$) and their eigenstates to be represented in any basis and for any
spin quantum number. Facilities for Clebsch-Gordan Coefficients, Wigner
Coefficients, rotations, and angular momentum coupling are also present in
their symbolic and numerical forms.

Second we have implemented a full set of states and operators for symbolic
quantum computing. Multidimensional qubit states can be represented
symbolically and as vectors. A full set of one ($X$, $Y$, $Z$, $H$, etc.) and
two qubit ($CNOT$, etc.) gates (unitary operators) are provided. These can be
represented as matricies (sparse or dense) or made to act on qubits
symbolically without representation. With these gates, it is possible to
implement a number of basic quantum circuits including the quantum Fourier
transform, quantum error correction, quantum teleportation, Grover's algorithm,
dense coding, etc.

Other examples of particular quantum systems that are implemented in SymPy
include second quantization, the simple harmonic oscillator (position/momentum
and raising/lowering forms) and continuous position/momentum based systems.

The package also contains exact symbolic energies and wave functions of several
simple systems like the Hydrogen atom (non-relativistic and relativistic) and
harmonic oscillator (1d and spherical 3D).

1. J.J. Sakurai and J.J. Napolitano, "Modern Quantum Mechanics," Addison-Wesley
(2010).
2. R.N. Zare, "Angular Momentum: Understanding Spatial Aspects in Chemistry and
Physics," Wiley (1991).
3. M.A. Nielsen and I.L. Chuang, "Quantum Computation and Quantum Information,"
Cambridge University Press (2011).



\subsection{Optics}

The \verb|physics.optics| package provides Gaussian optics functions.

% TODO : This needs some help from someone that knows something about optics.

\subsection{Units}

The \verb|physics.units| module provides around two hundred predefined prefixes
and SI units that are commonly used in the sciences. Additionally, it provides
the \verb|Unit| class which allows the user to define their own units.  These
prefixes and units are multiplied by standard SymPy objects to make expressions
unit aware, allowing for algebraic and calculus manipulations to be applied to
the expressions while the units are tracked in the manipulations.  The units of
the expressions can be easily converted to other desired units.  There is also
a new units system in \verb|sympy.physics.unitsystems| that allows the user to
work in specified unit systems.

\subsection{Tensors}

Ongoing work to provide the capabilities of tensor computer algebra has so far
produced the \verb|tensor| module.  It is composed of three separated
submodules, whose purposes are quite different: \verb|tensor.indexed| and
\verb|tensor.indexed_methods| support indexed symbols,
\verb|tensor.array| contains facilities to operator on symbolic $N$-dimensional
arrays and finally \verb|tensor.tensor| is used to defineabstract tensors.
The abstract tensors subsection
is inspired by xAct\cite{xAct} and Cadabra\cite{Peeters2007cadabra}.
Canonicalization based on the Butler-Portugal\cite{ManssurPortugal1999}
algorithm is supported in SymPy.  It is currently limited to polynomial tensor
expressions.

